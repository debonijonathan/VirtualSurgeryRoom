% Per presentazione
\documentclass{beamer}

%Per stampare senza animazioni:
%\documentclass[handout]{beamer}

% Per stampare più slides per pagina
%\usepackage{pgfpages}
%\pgfpagesuselayout{4 on 1}[a4paper, landscape, border shrink=5mm]		%vedi file presentazione_handout_4on1
%\pgfpagesuselayout{2 on 1}[a4paper, portrait, border shrink=5mm]		%vedi file presentazione_handout_2on1
%\nofiles


\usetheme{SUPSI}


\titolo{Titolo della presentazione}
\studente{Nome studente}
\relatore{Nome relatore}
\correlatore{Nome correlatore}
\committente{Nome committente}
\corso{Ing. informatica}
\opzione{???}
\anno{2012-2013}
% se si vuole una data fissa
%\data{February 13, 2013}
% Se si vuole la data corrente, mettere:
\data{ \today}



\begin{document}

\makeatletter
    \newenvironment{withoutfootline}{
        \setbeamertemplate{footline}[default]
        \def\beamer@entrycode{\vspace*{\footheight}}
    }{}
\makeatother


% Per aggiungere una pagina di titolo prima delle nuove sezioni
\AtBeginSection[] { 
  \begin{frame}
	\mysectionpage
  \end{frame} 
} 


\maketitle
\begin{frame}{Agenda}
   \tableofcontents
\end{frame}

\section{Sections, subsections}
\subsection{}
\begin{frame}
Le section generano una sezione nuova che appare nel TOC e nel footer.\\
Le subsection generano i puntini del footer. Il puntino colorato rappresenta la subsection corrente.
\end{frame}

\section{Elenchi e blocchi di testo}
\subsection{Elenchi puntati}
\begin{frame}{Elenco puntato senza animazioni}
\begin{itemize}
	\item item1
	\item item2
	\item item3
	\item item4
\end{itemize}
\end{frame}

\begin{frame}{Elenco puntato con apparizione animata}
\begin{itemize}
	\item<1-> item1
	\item<2-> item2
	\item<3-> item3
	\item<4-> item4
\end{itemize}
\end{frame}

\begin{frame}{Elenco puntato animato e con effetto trasparenza}
\setbeamercovered{transparent}
\begin{itemize}
	\item<1-> item1
	\item<2-> item2
	\begin{itemize}
		\item subitem1
		\item subitem2
	\end{itemize}
	\item<3-> item3
	\item<4-> item4
\end{itemize}
\end{frame}

\subsection{Blocchi di testo}
\begin{frame}{Blocchi di testo animati}
\visible<1->{
	Blocco 1 \\
}	
\visible<2->{
	Blocco 2
}	
\end{frame}

\begin{frame}{Blocchi di testo animati con trasparenza}
\setbeamercovered{transparent}
\uncover<1->{
	Blocco 1 \\
}	
\uncover<2->{
	Blocco 2
}
\end{frame}

\section{Spazi e testo in colonne}
\subsection{Spazi tra linee}
\begin{frame}{Spazi tra linee}
Prima linea di testo con bigskip\\
\bigskip
Seconda linea di testo con [4ex] \\[4ex]
Terza linea con vspace di baselineskip \\
\vspace{\baselineskip}
Quarta linea con vspace fisso a 0.5 cm \\
\vspace{0.5cm}
Fine
\end{frame}

\subsection{Testo in colonne}
\begin{frame}{Testo in colonne}
\begin{columns}
	\column{.4\textwidth}
	Testo nella colonna di sinistra, larga il 40\% dello spazio per il testo

	\column{.6\textwidth}
	Testo nella colonna di destra, larga il 60\% dello spazio per il testo
\end{columns}
\end{frame}

\section{Frame speciali}
\subsection{}
\begin{frame}[fragile]{Testo tipo codice, verbatim}
\begin{semiverbatim}
@RELATION iris
\small{
@ATTRIBUTE sepallength	REAL
@ATTRIBUTE sepalwidth	REAL
@ATTRIBUTE petallength	REAL
@ATTRIBUTE petalwidth	REAL
@ATTRIBUTE class \{Iris-setosa,Iris-versicolor,Iris-virginica\}} 

@DATA
5.1, 3.5, 1.4, 0.2, Iris-setosa
4.9, 3.0, 1.4, 0.2, Iris-setosa
7.0, 3.2, 4.7, 1.4, Iris-versicolor
6.4, 3.2, 4.5, 1.5, Iris-versicolor
6.3, 3.3, 6.0, 2.5, Iris-virginica
5.8, 2.7, 5.1, 1.9, Iris-virginica
\dots
\end{semiverbatim}
\end{frame}

\subsection{Frame senza footer}
\begin{withoutfootline}
\begin{frame}{Frame senza footer}
	Questo frame non ha la zona footer
\end{frame}
\end{withoutfootline}




\section{Math, img e tab}
\subsection{Formule matematiche}
\begin{frame}{Formule matematiche}
Con blocco equation:
\begin{equation}
mean = \frac{1}{n}\sum_{i=1}^n x_i
\end{equation}

Con blocco equation* (senza numero):
\begin{equation*}
mean = \frac{1}{n}\sum_{i=1}^n x_i
\end{equation*}

Con blocco math:
\begin{math}
mean = \frac{1}{n}\sum_{i=1}^n x_i
\end{math}

\end{frame}

\subsection{Immagini}
\begin{frame}{Immagine}
\begin{figure}[ht!]
\centering
\includegraphics[scale=0.3]{SUPSI.png}
\label{fig:logoSUPSI}
\caption{Logo SUPSI}			
\end{figure}
\end{frame}

\subsection{Tabelle}
\begin{frame}{Tabella}
\begin{table}[htp!]
\label{tab:results}
\caption{Tabella dei risutati}
\begin{center}
\begin{tabular}{lccc}
\hline
Algorithm 	& Scene & Yeast & Slashdot\\
\hline
BR-NB 	& 0.280  & 0.092  & 0.361 \\
ECC-NB 	& 0.300  & 0.101  & 0.368 	\\
ECC-J48 	& 0.525  & \textbf{0.128}  & 0.370 	\\
EBN-J 		& \textbf{0.561}  & 0.124  & \textbf{0.385}	\\
\hline
\end{tabular}
\end{center}
\end{table}
\end{frame}

\section{Alert}
\subsection{Evidenziare testo con alert}
\begin{frame}{Evidenziare testo con alert}
\visible<1->{Quanto fa 2+2?  \{\alert<2->{4}, 5\}.}
\end{frame}

\end{document}